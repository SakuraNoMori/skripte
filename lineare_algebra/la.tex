\documentclass[a4paper,10pt]{scrartcl}
\usepackage{amsmath}
\usepackage{graphicx}
\usepackage{hyperref}
\usepackage[german]{babel}
\usepackage[utf8]{inputenc}

\title{Lineare Algebra Skript}
\author{Arif Hasanic}

\begin{document}
    \maketitle
    \newpage
    \tableofcontents
    \newpage

    \section{Einleitung}
        \subsection{Logik}
        \subsubsection{Aussagenlogik}
        Die Aussagenlogik beschreibt einen Sachverherhalt, dem man eindeutig einen 
        Wahrheitswert (wahr, falsch) zuordnen kann. Weiter kann man diese Ausdrücke verknüpfen. 
        z = x $\wedge$ y,   x und y müssen beide wahr sein damit z wahr ist.\footnote{x bzw. y sind Platzhalter. Beispielsweise könnte für x: 1 > 2 stehen} \\
        Die Erfüllungsmenge eines aussagenlogischen Ausdrucks besteht aus allen Variablen $x_i$ für die der gesamte Ausdrucke wahr ist
        \begin{description}
            \item[$z_1 \Rightarrow z_2$]
            \item $z_2$ ist notwendig für $z_1$ 
            \item $z_1$ ist hinreichend für $z_2$
        \end{description}

        \subsubsection{Prädikatenlogik}
        Bei der Prädikatenlogik wird eine Aussage in Subjekt und Prädikt aufgeteilt. Das Subjekt dient als Platzhalter. Der Vorteil ist, dass nun 
        allgemeinere Aussagen erstellt werden können. Beispiel: "s studiert Maschinenbau", wobei s $\epsilon$ Studenten. \\
        Die Ergebnismenge besteht dann aus den Aussagen die zutreffen. Prädikate können wie in der Aussagenlogik verknüpft werden und außerdem werden noch sogenannte
        Quantoren eingführt:
        \begin{description}
            \item $\forall$: Der Allqunator sagt aus dass  Prädikate für alle Elemte der einer Menger gelten \\($\forall$ s $\epsilon$ Studenten)
            \item $\exists$: Der Existenzquantor Prädikat für mindestens ein Element der Menge wahr ist.
            \item $!\exists$: Dieser Qunator bedeutet, dass das Prädikat für genau ein Subjekt (Element aus Menge) gilt.
        \end{description} 
        \subsection{Mengen}
        \subsection{Realtionen}
        \subsection{Induktion}

    \section{Lineare Gleichungssysteme}
        \subsection{Einführung}
        \subsection{LGS lösen}
        \subsection{LGS aufstellen}
        \subsection{Determinanten}
        \subsection{Matrizen}
    
    \section{Vektoren}
        \subsection{Koordinatensysteme}
        \subsection{Rechenoperationen}
        \subsection{Gerade, Ebene}
        \subsection{Skalarprodukt}
        \subsection{Vektorprodukt}
    
    \section{Gruppen, Körper, Vektorräume}
        \subsection{Gruppen}
        \subsection{Körper}
        \subsection{Vektorraum}
        \subsection{Basis}

    \section{Lineare Abbildungen}
        \subsection{Definition}
        \subsection{Darstellung durch Matrix}
        \subsection{Rechenoperationen}
        \subsection{Eigenvektor und Eigenwert}
    
\end{document}
    
