\documentclass[a4paper,10pt]{scrartcl}
\usepackage{amsmath}
\usepackage{graphicx}
\usepackage{hyperref}
\usepackage[utf8]{inputenc}

\title{Netzwerke}
\author{}
\date{}

\begin{document}
\maketitle
\newpage
\tableofcontents
\newpage
\section{Kapitel}
    \subsection{DNS - Domain Name System}
        Ein Domain Name System nimmt Internetadressen wie 'facebook.com' und liefert dessen IP-Adresse, 
        damit Rechenr sich damit verbinden können. Wobei 'com'\footnote[1]{Eigentlich ist rechts von einer Domain ein Punkt 'com.', dies wird aber normalerweise nie geschrieben} eine TLD (Top Level Domain) ist. Man unterscheidet TLDs wie folgt:
        \begin{description}
            \item [gTLD] (auch: genericTLDs, allgemeine TLD) Diese werden wieder in 2 Untergruppen aufgeteilt:
                \begin{description}
                    \item [sTLD] (auch: sponsored TLD) Diese TLD werden nur an Websiten vergeben, welche bestimmte Forderungen erfüllen. '.gov'
                    \item [uTLD] (auch: unsponsered TLD) TLD werden ohne Vorgaben vergeben. '.com, .xyz' 
                \end{description} 
            \item [ccTLD] (auch: country-codeTLD) TLD die zeigen aus welchem Land die Website kommt. '.de oder .us'
        \end{description}
        Beim Beispiel von facebook.com nennt man das .facebook eine Second-Level Domain, würde da noch www. stehen wäre das die Third-Level Doamin,
        Es können (quasi) beliebig viele Subdomains eingeführt werden. Die niedrigste Subdomain heißt hierbei Lowest-Level Domain. \\
        Eine FQDN (Fully Qualieified Domain Name) setzt sich aus Top, Lowest und min. einer Domain dazwischen zusammem.
    \subsection{LAN - Local Area Network}
\end{document}
